\documentclass[a4paper,10pt,twocolumn]{article}

\usepackage[utf8]{inputenc}
\usepackage{amsmath}
\usepackage{graphicx}
\usepackage{hyperref}
\usepackage{float}
\usepackage[margin=0.5in]{geometry}
\usepackage{subcaption}

\title{OR: Paper Review \\ \Large{MaGIC: Multi-modality Guided Image Completion~\cite{magic}}}
\author{Bruno Sánchez Gómez}
\date{\today}

\begin{document}

\twocolumn[
\maketitle
\begin{abstract}
    This report presents a comprehensive reproduction of the Path Integral Clustering (PIC) algorithm introduced by Zhang et al \cite{PIC}. PIC leverages concepts from statistical physics to effectively cluster data with complex manifold structures by measuring connectivity through path integrals. We implement the algorithm and evaluate it against eleven state-of-the-art clustering methods on both synthetic and real-world imagery datasets (MNIST, USPS, and Caltech-256). Our experiments confirm PIC's exceptional performance, achieving the highest Normalized Mutual Information scores on MNIST ($0.940$) and Caltech-256 ($0.653$), while showing competitive results on USPS. We extend the evaluation to include additional datasets (Iris and Breast Cancer) and metrics (Silhouette score), providing insights into PIC's strengths with non-convex clusters. Scalability analysis reveals that PIC's runtime increases primarily with sample count rather than dimensionality. Despite some reproduction challenges due to implementation ambiguities, our results validate PIC as a significant advancement in clustering methodology, particularly for datasets with complex structures.
\end{abstract}
\vspace{2em}
]

\input{sections/introduction}

\input{sections/pic}

\input{sections/experiments}

\input{sections/conclusion}

\bibliographystyle{unsrt}
\bibliography{references}

\end{document}