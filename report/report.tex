\documentclass{beamer}
\usepackage[utf8]{inputenc}
\usepackage{cite}
\usepackage{graphicx}
\usepackage{float}
\usepackage{fontawesome}
\usepackage{multicol}
\usepackage{listings}
\usepackage{array}
\usepackage{subcaption}


% Beamer theme settings
\usetheme{Madrid}
\usecolortheme{default}

\title{OR Paper Review \\MaGIC:  Multi-modality Guided Image Completion}
\author{Bruno Sánchez Gómez}
\date{\today}

\setbeamertemplate{footline}{%
    \leavevmode%
    \hbox{%
        \begin{beamercolorbox}[wd=0.2\paperwidth,ht=2.5ex,dp=1.125ex,center]{author in head/foot}%
            \hspace*{2mm}Bruno Sánchez Gómez
        \end{beamercolorbox}%
        \begin{beamercolorbox}[wd=0.7\paperwidth,ht=2.5ex,dp=1.125ex,center]{title in head/foot}%
            \insertshorttitle
        \end{beamercolorbox}%
        \begin{beamercolorbox}[wd=0.1\paperwidth,ht=2.5ex,dp=1.125ex,center]{date in head/foot}%
            \hspace*{-2mm}\insertframenumber{} / \inserttotalframenumber
        \end{beamercolorbox}%
    }%
    \vskip0pt%
}

\begin{document}

\begin{frame}
    \titlepage
\end{frame}


\begin{frame}{Table of Contents}
    \tableofcontents
\end{frame}

\section{MaGIC Overview}
\begin{frame}{Multi-modal Guided Image Completion (MaGIC)}
    \begin{itemize}
        \item \textbf{Problem:} Vanilla image completion struggles with large missing regions; existing guided methods often use only a single modality.
        \item \textbf{MaGIC Solution:} A flexible framework for image completion guided by single or \textit{arbitrary combinations} of modalities (Text, Edge, Sketch, Segmentation, Depth, Pose).
        \item \textbf{Components:}
        \begin{itemize}
            \item \textbf{Modality-specific Conditional U-Net (MCU-Net):} Injects single-modal guidance into a U-Net denoiser.
            \item \textbf{Consistent Modality Blending (CMB):} Training-free method to blend guidance from multiple pre-trained MCU-Nets via latent space gradients. Enables easy addition of new modalities.
        \end{itemize}
        \item \textbf{Results:} Outperforms SOTA methods and generalizes well to various completion tasks.
    \end{itemize}
\end{frame}

\begin{frame}{Examples}
    Image examples from the paper (maybe more than 1 slide)
\end{frame}

\begin{frame}{MCU-Net}
    Explain MCU-Net
\end{frame}

\begin{frame}{CMB}
    Explain CMB
\end{frame}

\section{Critical Analysis}
\begin{frame}{Why did MaGIC succeed?}
    
\end{frame}

\begin{frame}{Where did MaGIC fail?}

\end{frame}

\begin{frame}{Future Implications [Placeholder]}
    \begin{itemize}
        \item \textbf{Generalization:} MaGIC's framework can be applied to other image generation tasks, such as inpainting or super-resolution.
        \item \textbf{Modality Fusion:} The CMB method can be extended to fuse more complex modalities, such as audio or video.
        \item \textbf{Real-world Applications:} Potential applications in fields like medical imaging, autonomous driving, and augmented reality.
    \end{itemize}
    
\end{frame}

\begin{frame}{References}
    \begin{itemize}
        \item \textbf{MaGIC:  Multi-modality Guided Image Completion} \\
        \url{https://arxiv.org/abs/2303.14100}
    \end{itemize}
\end{frame}

\end{document}