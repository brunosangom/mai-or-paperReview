\documentclass[pdf]{beamer}
\usepackage[utf8]{inputenc}
\usepackage{cite}
\usepackage{graphicx}
\usepackage{float}
\usepackage{fontawesome}
\usepackage{multicol}
\usepackage{listings}
\usepackage{array}
\usepackage{subcaption}


% Beamer theme settings
\usetheme{Madrid}
\usecolortheme{default}

\title{OR Paper Review \\MaGIC:  Multi-modality Guided Image Completion}
\author{Bruno Sánchez Gómez}
\date{\today}

\setbeamertemplate{footline}{%
    \leavevmode%
    \hbox{%
        \begin{beamercolorbox}[wd=0.2\paperwidth,ht=2.5ex,dp=1.125ex,center]{author in head/foot}%
            \hspace*{2mm}Bruno Sánchez Gómez
        \end{beamercolorbox}%
        \begin{beamercolorbox}[wd=0.7\paperwidth,ht=2.5ex,dp=1.125ex,center]{title in head/foot}%
            \insertshorttitle
        \end{beamercolorbox}%
        \begin{beamercolorbox}[wd=0.1\paperwidth,ht=2.5ex,dp=1.125ex,center]{date in head/foot}%
            \hspace*{-2mm}\insertframenumber{} / \inserttotalframenumber
        \end{beamercolorbox}%
    }%
    \vskip0pt%
}

\begin{document}

\begin{frame}
    \titlepage
\end{frame}


\begin{frame}{Table of Contents}
    \tableofcontents
\end{frame}

\section{Introduction}

\begin{frame}{Image Completion}
    \centering
    \begin{minipage}{0.8\textwidth}
        \begin{block}{Definition}
            \centering
            \textit{Image completion} refers to the task of filling in missing regions within an image in a visually plausible way.
        \end{block}
    \end{minipage}
    \vspace{0.5cm}
    \begin{itemize}
        \item \textbf{Applications:}
            \begin{itemize}
                \item \textbf{Inpainting:} Restoring damaged or missing parts of an image.
                \item \textbf{Outpainting:} Extending the boundaries of an image.
                \item \textbf{Editing:} Modifying images by adding or removing elements.
            \end{itemize}
    \end{itemize}
\end{frame}

\begin{frame}{Approaches to Image Completion}
    \begin{itemize}
        \item \textbf{Vanilla Image Completion:}
        \begin{itemize}
            \item Relies solely on existing image pixels around the masked region.
            \item Struggles with large missing areas due to limited internal context.
            \item Often leads to blurry or repetitive textures.
        \end{itemize}
        \item \textbf{Guided Image Completion:}
        \begin{itemize}
            \item Uses external cues (e.g., text descriptions, edge maps, segmentation masks) for guidance.
            \item Improves results significantly, especially for large gaps, by providing external semantic information.
            \item Existing methods often restricted to \textit{single-modality} guidance, limiting flexibility and performance in complex scenarios requiring multiple constraints.
        \end{itemize}
    \end{itemize}
\end{frame}

\begin{frame}{Multi-modal Guided Image Completion (MaGIC)}
    \begin{itemize}
        \item \textbf{MaGIC:} A flexible framework for image completion guided by single or \textit{arbitrary combinations} of modalities, such as:
        \begin{itemize}
            \item Text
            \item Canny Edge
            \item Sketch
            \item Segmentation
            \item Depth
            \item Pose
        \end{itemize}
        \item \textbf{Results:} Outperforms SOTA methods and generalizes well to various completion tasks.
    \end{itemize}
\end{frame}

\section{MaGIC Overview}
\begin{frame}{Table of Contents}
    \tableofcontents[currentsection]
\end{frame}
\begin{frame}{Examples}
    Image examples from the paper (maybe more than 1 slide)
\end{frame}

\begin{frame}{Components of MaGIC}
    \begin{itemize}
        \item \textbf{Modality-specific Conditional U-Net (MCU-Net):} Injects single-modal guidance into a U-Net denoiser.
        \item \textbf{Consistent Modality Blending (CMB):} Training-free method to blend guidance from multiple pre-trained MCU-Nets via latent space gradients. Enables easy addition of new modalities.
    \end{itemize}
\end{frame}

\begin{frame}{MCU-Net}
    Explain MCU-Net
\end{frame}

\begin{frame}{CMB}
    Explain CMB
\end{frame}

\section{Critical Analysis}
\begin{frame}{Table of Contents}
    \tableofcontents[currentsection]
\end{frame}
\begin{frame}{Why did MaGIC succeed?}
    
\end{frame}

\begin{frame}{Where did MaGIC fail?}

\end{frame}

\begin{frame}{Future Implications [Placeholder]}
    \begin{itemize}
        \item \textbf{Generalization:} MaGIC's framework can be applied to other image generation tasks, such as inpainting or super-resolution.
        \item \textbf{Modality Fusion:} The CMB method can be extended to fuse more complex modalities, such as audio or video.
        \item \textbf{Real-world Applications:} Potential applications in fields like medical imaging, autonomous driving, and augmented reality.
    \end{itemize}
    
\end{frame}

\begin{frame}{References}
    \begin{itemize}
        \item \textbf{MaGIC:  Multi-modality Guided Image Completion} \\
        \url{https://arxiv.org/abs/2303.14100}
    \end{itemize}
\end{frame}

\end{document}